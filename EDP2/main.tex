\documentclass[french,a4paper,10pt]{article}
\makeatletter % Allows the use of @ in command names

% Font and Language Settings ----------------------------------------------------
\usepackage[T1]{fontenc} % Use T1 font encoding for better character representation
\usepackage[french]{babel} % Set document language to French
\usepackage{lmodern} % Use Latin Modern font

% List Customization ------------------------------------------------------------
\usepackage[shortlabels]{enumitem} % Enhanced control over lists
\setlist[itemize,1]{label={\color{gray}\small \textbullet}} % Customize the bullet style for itemize

% Page subpart Customization -----------------------------------------------
\usepackage{fancyhdr} % Create custom headers and footers
\usepackage{titlesec} % Customize section titles
\usepackage{titletoc} % For better TOC control

% Mathematical Symbols and Enhancements -----------------------------------------
\usepackage{centernot} % Provides the \centernot command for centered negation
\usepackage{stmaryrd} % Provides extra mathematical symbols like \llbracket
\usepackage[overload]{abraces} % Provides extensible braces
\usepackage{latexsym} % Standard LaTeX symbols
\usepackage{amsmath} % American Mathematical Society mathematical features

\usepackage{amsfonts} % AMS font package
\usepackage{amssymb} % Additional AMS symbols
\usepackage{amsthm} % Theorem environments
\usepackage{mathtools} % Mathematical tools to extend amsmath
\usepackage{mathrsfs} % Provides \mathscr for script letters
\usepackage{MnSymbol} % More math symbols
\usepackage{etoolbox} % Conditional macros
\usepackage{hyperref} % Hyperlinks within the document
\usepackage{bigints}

% Graphing Tools ----------------------------------------------------------------
\usepackage{tikz} % Create graphics programmatically
\usepackage{pgfplots} % Create plots using TikZ
\pgfplotsset{compat=1.18} % Set compatibility level for pgfplots
\usetikzlibrary{arrows} % Add arrow tips to TikZ

% Color Customization -----------------------------------------------------------
\usepackage{xcolor} % Color management
\usepackage{colortbl} % Color management for tables
\definecolor{astral}{RGB}{46,116,181} % Define astral color
\definecolor{verdant}{RGB}{96,172,128} % Define verdant color
\definecolor{algebraic-amber}{RGB}{255,179,102} % Define algebraic-amber color
\definecolor{calculus-coral}{RGB}{255,191,191} % Define calculus-coral color
\definecolor{divergent-denim}{RGB}{130,172,211} % Define divergent-denim color
\definecolor{matrix-mist}{RGB}{204,204,204} % Define matrix-mist color
\definecolor{numeric-navy}{RGB}{204,204,204} % Define numeric-navy color
\definecolor{quadratic-quartz}{RGB}{204,153,153} % Define quadratic-quartz color

% Custom Theorem Styles ---------------------------------------------------------
\usepackage[]{mdframed} % Framing for environments

% Custom sectioning environments ------------------------------------------------

% Set up counters
\setcounter{secnumdepth}{5}
\setcounter{tocdepth}{5}

% Define suprasection
\titleclass{\suprasection}{straight}[\part]
\newcounter{suprasection}
\renewcommand\thesuprasection{\arabic{suprasection}}

% Format the suprasection (in document) with matching style
\titleformat{\suprasection}
  {\color{astral}\normalfont\sffamily\bfseries\Large\filcenter}{\thesuprasection}{1em}{}
\titlespacing*{\suprasection}{0pt}{3.5ex plus 1ex minus .2ex}{2.3ex plus .2ex}

% Adjust TOC formatting - shift everything down one level
\dottedcontents{suprasection}[1.5em]{}{2.3em}{1pc}  % Like a normal section
\dottedcontents{section}[3.8em]{}{2.8em}{1pc}       % Like a normal subsection
\dottedcontents{subsection}[7.0em]{}{3.2em}{1pc}    % Like a normal subsubsection
\dottedcontents{subsubsection}[10.0em]{}{3.8em}{1pc}
% Add to TOC
\titlecontents{suprasection}
  [0em]  % Less indentation than regular sections
  {\addvspace{1pc}\bfseries}  % Above code (bold, extra space)
  {\contentslabel{2em}}
  {\hspace*{-2em}}
  {\titlerule*[0.5pc]{.}\contentspage}


% Redeclare math operators for customization ------------------------------------
\makeatletter
\newcommand\RedeclareMathOperator{%
	\@ifstar{\def\rmo@s{m}\rmo@redeclare}{\def\rmo@s{o}\rmo@redeclare}%
}
\newcommand\rmo@redeclare[2]{%
	\begingroup \escapechar\m@ne\xdef\@gtempa{{\string#1}}\endgroup
	\expandafter\@ifundefined\@gtempa
	{\@latex@error{\noexpand#1undefined}\@ehc}%
	\relax
	\expandafter\rmo@declmathop\rmo@s{#1}{#2}}

\newcommand\rmo@declmathop[3]{%
	\DeclareRobustCommand{#2}{\qopname\newmcodes@#1{#3}}%
}
\@onlypreamble\RedeclareMathOperator
\makeatother


% Miscellaneous Commands --------------------------------------------------------
\setlength{\parindent}{0pt} % Remove paragraph indentation

% Define emphasis in text
\providecommand{\defemph}[1]{{\sffamily\bfseries\color{astral}#1}}

% Section styling
\usepackage{sectsty} % Allows customizing section fonts
\allsectionsfont{\color{astral}\normalfont\sffamily\bfseries} % Style all section headers

\newcommand{\skipline}{\vspace{\baselineskip}} % Skip a line
\newcommand{\noi}{\noindent} % No indentation

\newcommand{\ptr}{\( \triangleright \)} % Triangle right symbol

\makeatother % End of @ usage
\makeatletter
% Custom Commands for fast mathematical notation --------------------------------
% General shortcuts
\newcommand{\scr}[1]{\mathscr{#1}} % Script font shortcut
\newcommand{\scrF}{\mathscr{F}} % Script font shortcut
\newcommand{\bb}[1]{\mathbb{#1}} % Blackboard bold shortcut
\newcommand{\ol}[1]{\overline{#1}} % Overline shortcut
\newcommand{\ul}[1]{\underline{#1}} % Underline shortcut
\newcommand{\act}{\circlearrowleft} % Action symbol
\newcommand{\vphi}{\varphi} 
\newcommand{\cxi}{\langle\xi\rangle} % <xi> poids pour les Sobolevs
\newcommand{\Dx}{\Delta x}
\newcommand{\Dy}{\Delta y}
\newcommand{\Uij}{U_{ij}}
\newcommand{\omb}{\overline{\Omega}} % Adhérence d'Omega
\newcommand{\eqxi}{(1+\n{\xi}^2)} % écrit (1 + ∣ξ∣^2) pour les Sobolevs
\newcommand{\RRd}{(\R \times \R^d)}
\newcommand{\et}{e^{t\triangle}}
\newcommand{\ijn}{(i,j)\in}

% Interval notation
\newcommand{\oo}[1]{\mathopen{}\left]#1\right[\mathclose{}} % Open interval % chktex 9
\newcommand{\ooo}{\mathopen{}\left]0,T\right[\mathclose{}} % Open interval % chktex 9
\newcommand{\of}[1]{\mathopen{}\left]#1\right]\mathclose{}} % Half-open interval (open first) % chktex 9 % chktex 10
\newcommand{\fo}[1]{\mathopen{}\left[#1\right[\mathclose{}} % Half-open interval (open last)
\newcommand{\ff}[1]{\mathopen{}\left[#1\right]\mathclose{}} % Closed interval % chktex 9
\newcommand{\ffo}{\mathopen{}\left[0,T\right]\mathclose{}} % Closed interval % chktex 9

% Set notation
\newcommand{\R}{\mathbb{R}} % Real numbers
\newcommand{\Z}{\mathbb{Z}} % Integers
\newcommand{\N}{\mathbb{N}} % Natural numbers
\newcommand{\C}{\mathscr{C}} % Fonctions continues
\newcommand{\Q}{\mathbb{Q}} % Rational numbers
\newcommand{\D}{\mathscr{D}} % Ensemble des fonctions tests
\newcommand{\Dp}{\mathscr{D^{'}}} % Ensemble des distributions 
\newcommand{\Sp}{\mathscr{S}^{'}} % Distributions tempérées
\newcommand{\scrS}{\mathscr{S}} % Classe de Schwartz
\newcommand{\Hs}{H^s(\R^d)}
\newcommand{\bbR}{\mathbb{R}} % Real numbers
\newcommand{\bbZ}{\mathbb{Z}} % Integers
\newcommand{\bbN}{\mathbb{N}} % Natural numbers
\newcommand{\bbC}{\mathbb{C}} % Complex numbers
\newcommand{\bbQ}{\mathbb{Q}} % Complex numbers
\newcommand{\scrP}{\mathscr{P}} % Set of subsets
\newcommand{\barR}{\overline{\bb{R}}} % R with infinities
\newcommand{\gln}{\text{GL}_n} % General linear group of degree n
\newcommand{\glx}[1]{\text{GL}_{#1}} % General linear group
%
\newcommand{\sub}{\subset} % Subset
\newcommand{\cequiv}[1]{\mathopen{}[#1\mathclose{}]} % Equivalence class
\newcommand{\restr}[2]{#1\mathop{}\!|_{#2}} % Restriction
\newcommand{\adh}[1]{\mathring{#1}} % Adherence
\newcommand{\Adh}[1]{\mathring{\overbrace{#1}}} % Big adherence
\newcommand{\comp}[1]{{#1}^C} % Complementary of a set

% Differential notation
\newcommand{\der}{\mathop{}\!{d}} % Differential operator
\newcommand{\p}{\mathop{}\!{\partial}} % Partial derivative operator
\providecommand{\dpar}[2]{\frac{\partial{#1}}{\partial{#2}}}

% Topology notation
\newcommand{\bolo}[1]{B\left({#1}\mathopen{}\right[\mathclose{}} % Open ball
\newcommand{\bolf}[1]{B\left({#1}\mathopen{}\right]\mathclose{}} % Closed ball

% Limits notation
\newcommand{\limi}{\underline{\lim}}
\newcommand{\lims}{\overline{\lim}}

% Norm notation
\newcommand{\norm}{\mathcal{N}} % Norm
\newcommand{\nn}[1]{\mathopen{}\left\|#1\right\|\mathclose{}} % Double bar norm
\newcommand{\nnn}[1]{\mathopen{}\left\||#1|\right\|\mathclose{}} % Double bar norm
\newcommand{\n}[1]{\mathopen{}\left|#1\right|\mathclose{}} % Single bar norm


% Other operators
\providecommand{\1}{\mathds{1}} % Identity operator
\DeclareMathOperator{\im}{\mathsf{Im}} % Imaginary part
\DeclareRobustCommand{\re}{\mathsf{Re}} % Real part
\DeclareMathOperator{\vect}{\mathsf{Vect}} % Vector space
\DeclareMathOperator{\diam}{\mathsf{Diam}} % Diameter
\DeclareMathOperator{\orb}{\mathsf{orb}} % Orbit
\DeclareMathOperator{\st}{\mathsf{st}} % Standard part
\DeclareMathOperator{\spr}{\mathsf{SP_{\bb{R}}}} % Real spectrum
\DeclareMathOperator{\aut}{\mathsf{Aut}} % Automorphism group
\DeclareMathOperator{\bij}{\mathsf{Bij}} % Bijection group
\DeclareMathOperator{\rank}{\mathsf{rank}} % Rank
\DeclareMathOperator{\tr}{\mathsf{tr}} % Trace
\DeclareMathOperator{\id}{\mathsf{Id}} % Identity
\DeclareMathOperator{\var}{\mathsf{Var}} % Variance
\DeclareMathOperator{\cov}{\mathsf{Cov}} % Covariance
\DeclareFontFamily{U}{mathx}{}
\DeclareFontShape{U}{mathx}{m}{n}{<-> mathx10}{}
\DeclareSymbolFont{mathx}{U}{mathx}{m}{n}
\DeclareMathAccent{\widecheck}{0}{mathx}{"71}
\providecommand{\B}{\mathsf{B}} % Bold symbol

\usepackage{dsfont}
\newcommand{\one}{\mathds{1}}

\newcommand{\smol}[1]{\text{\scriptsize{#1}}}

\newcommand{\bleu}[1]{{\color{blue} #1}}
\newcommand{\ver}[1]{{\color{green} #1}}
\newcommand{\rouge}[1]{{\color{red}#1}}
\newcommand{\orange}[1]{{\color{orange}#1}}



\makeatother

\newtheoremstyle{default}{\topsep}{\topsep}%
{}% Body font
{}% Indent amount (empty = no indent, \parindent = para indent)
{\sffamily\bfseries}% Thm head font
{.}% Punctuation after thm head
{ }% Space after thm head (\newline = linebreak)
{\thmname{#1}\thmnumber{~#2}\thmnote{~#3}}% Thm head spec

\newtheoremstyle{nonum}{\topsep}{\topsep}%
{}%
{}%
{\sffamily\bfseries}%
{.}%
{ }%
{\thmname{#1}}%

\newcommand{\mytheorem}[5]{%
	\ifstrequal{#5}{o}{%
		\newmdtheoremenv[
		hidealllines=true,
		skipabove=0pt,
		innertopmargin=-5pt,
		innerbottommargin=2pt,
		linewidth=1pt,
        innerleftmargin=0pt,
		]{#1}[#4]{#2}%
	}{% Dependant counter
		\newmdtheoremenv[
		hidealllines=true,
		skipabove=0pt,
		innertopmargin=-5pt,
		innerbottommargin=2pt,
		linewidth=1pt,
		linecolor=#3,
        innerleftmargin=0pt,
		]{#1}{#2}[#4]%
	}%
}

% Custom Global Theorems --------------------------------------------------------
\theoremstyle{default}
\newtoggle{showsolutions}
\newcounter{oc-counter}
\mytheorem{oc-proposition}{Proposition}{divergent-denim}{oc-counter}{o}
\mytheorem{oc-propdef}{Proposition - Définition}{divergent-denim}{oc-counter}{o}
\mytheorem{oc-theorem}{Théorème}{divergent-denim}{oc-counter}{o}
\mytheorem{oc-lemme}{Lemme}{quadratic-quartz}{oc-counter}{o}
\mytheorem{oc-example}{Exemple}{quadratic-quartz}{oc-counter}{o}
\mytheorem{oc-remark}{Remarque}{matrix-mist}{oc-counter}{o}
\mytheorem{oc-exercise}{Exercice}{calculus-coral}{oc-counter}{o}
\mytheorem{oc-definition}{Definition}{algebraic-amber}{oc-counter}{o}
\mytheorem{oc-corollaire}{Corollaire}{violet}{oc-counter}{o}
\counterwithin{oc-counter}{subsection}



\makeatletter


%--------------------------------------------------------------------------------
\newcommand{\cv}{\longrightarrow}
\newcommand{\cvn}{\underset{n\to\infty}{\cv}}
\newcommand{\cvps}{{\overset{\text{p.s.}}{\longrightarrow}}}
\newcommand{\cvpsn}{\underset{n\to\infty}{\cvps}}
\newcommand{\cvp}{{\overset{\mathbb{P}}{\longrightarrow}}}
\newcommand{\cvpn}{\underset{n\to\infty}{\cvp}}
\newcommand{\cvlp}{{\overset{\mathcal{L}^{p}}{\longrightarrow}}}
\newcommand{\cvlpn}{\underset{n\to\infty}{\cvlp}}
\newcommand{\cvd}{{\overset{(\mathcal{D})}{\longrightarrow}}}
\newcommand{\cvdn}{\underset{n\to\infty}{\cvd}}
\newcommand{\cvlaw}{{\overset{(\mathscr{L})}{\longrightarrow}}}
\newcommand{\cvlawn}{\underset{n\to\infty}{\cvlaw}}
\newcommand{\iid}{\text{i.i.d.}}




%--------------------------------------------------------------------------------
% Custom Per Subject Theorems
%--------------------------------------------------------------------------------

% -- Numerotation classique des theoremes
\theoremstyle{default}
\mytheorem{definition}{Définition}{algebraic-amber}{}{o}
\mytheorem{exs}{Exemples}{matrix-mist}{}{o}
\mytheorem{corollaire}{Corollaire}{violet}{}{o}
\mytheorem{lemme}{Lemme}{}{}{o}
\mytheorem{proposition}{Proposition}{verdant}{}{o}
\mytheorem{theorem}{Théorème}{astral}{}{o}

% -- Theoremes sans numeros -- %
\theoremstyle{nonum}
\mytheorem{nodefinition}{Définition}{algebraic-amber}{}{o}
\mytheorem{noproposition}{Proposition}{verdant}{}{o}
\mytheorem{noexample}{Exemple}{matrix-mist}{}{o}
\mytheorem{notheorem}{Théorème}{astral}{}{o}
\mytheorem{norappel}{Rappel}{red}{}{o}
\mytheorem{noremark}{Remarque}{}{}{o}
\mytheorem{nopropriete}{Propriété}{}{}{o}
\mytheorem{nonotation}{Notation}{}{}{o}
\makeatother
\input{figures.tex}
\usepackage[a4paper,hmargin=30mm,vmargin=30mm]{geometry}

\title{\color{astral} \sffamily \bfseries Analyse des EDP 2 -- Cours}
\author{Raphaël Bigey}


\begin{document}
    \maketitle
    \tableofcontents

    \newpage
    \begin{center}
        \section*{Chapitre 1 --- Equation de transport}\label{sec:CH1}
    \end{center}
    
    \addcontentsline{toc}{section}{Chapitre 1 --- Equation de transport}
    \setcounter{section}{1}
    \setcounter{subsection}{0}
    
        \subsection{Equation de transport à coefficients constants}\label{subsec:1.1}
        
            Soit $v \in \R^d$ un vecteur non nul. On s'intéresse ici à l'étude de l'équation de transport (parfois nommée équation d'advection, ou de convection).
    
            \[
            \partial_t u(t,x) + v \cdot \nabla u(t,x) = 0
            \]
            et plus particulièrement à la recherche de solutions régulières (nommées aussi \(\textbf{solutions fortes}\))
    
            \begin{definition}[Courbe caractéristique] \label{def:1.1.1}~
            
                Soit \(z \in \R \). On appelle \defemph{courbe caractéristique issue de $z$} de l'opérateur \( \partial_t + v \cdot \nabla\) l'ensemble \( \{(t,y(t)), t \in \R\}\) où $y$ est solution du problème de Cauchy 
                \[
                \begin{cases}
                    \frac{dy}{dt}(t) & = v \\
                    y(0) & = z
                \end{cases}
                \]           
                
            \end{definition}
    
            \begin{remark}\label{rem:1.1.2}
                Clairement, puisque $v$ est constant, on a dans ce cas \(y(t) = z+tv\) pour tout $t \in \R$. On déduit que la courbe caractéristique $\{(t,z+tv),t\in\R\}$ est donc une droite de $\R \times \R^d$.
            \end{remark}
    
            \begin{proposition}\label{prop:1.1.3}
                Soit \( u \in \mathscr{C}^1(\R_+ \times \mathbb{R}^d) \) une solution de l'équation de transport 
                \[
                \partial_t u(t,x) +v \cdot \nabla u(t,x) = 0.
                \]
                Soit \( z \in \R^d \) et \( \{ (t, y(t)), t \in \mathbb{R}_+ \} \) la courbe caractéristique issue de \( z \). \\
                Alors l'application \( t \mapsto u(t, y(t)) \) est de classe \( \mathscr{C}^1 \) sur \( \mathbb{R}_+ \), et \( u \) est constante le long de l'ensemble \( \{ (t, y(t)), t \in \mathbb{R}_+ \} \).
            \end{proposition}
        
            \begin{proof}
                ...
            \end{proof}
            
            \begin{theorem}\label{thm:1.1.4}
                Soit $v \in \R$ et $u_0 \in \C^1(\R^d)$. Le problème de Cauchy
                \[
                \begin{cases}
                    \partial_t u(t,x) + v \cdot \nabla u(t,x)  = 0 & (t,x) \in \R^*_+ \times \R^d \\
                    u(0,x)  = u_0(x) &x \in \R^d
                \end{cases}
                \]
                admet une unique solution $u \in \C^1(\R \times \R^d)$, donnée par la formule 
                \[
                \forall (t,x) \in \R_+ \times \R^d, u(t,x) = u_0(x-tv)
                \]
            \end{theorem}
            
            \begin{proof}
                ...
            \end{proof}
    
        \subsection{Equation de transport à coefficients variables}\label{subsec:1.2}
        
            On considère dans la suite un champ de vecteur $b : [0,T]\times \R^d \xrightarrow{} \R^d$, pour $T>0$ fixé.
            On s'intéresse ici au problème de Cauchy 
            \[
            \begin{cases}
                \partial_t u(t,x) + b(t,x) \cdot \nabla u(t,x)  = 0 & (t,x) \in \oo{0,T} \times \R^d \\
                u(0,x)  = u_0(x) &x \in \R^d
            \end{cases}
            \]
            On suppose que $b$ admet des dérivées partielles d'ordre 1 par rapport à $x_i, i\in \llbracket 1, d \rrbracket$, et qu'il satisfait les conditions suivantes:

           
            \begin{enumerate}
                \item $b \in \C^0(\ff{0,T}\times \R^d)$
                
                \item $\nabla b \in \C^0(\ff{0,T}\times \R^d)$

                \item $\exists C_b>0, \forall (t,x)\in \ff{0,T} \times \R^d, \n{b(t,x)} \leq C_b(1+\n{x})$ ("croissance lente")
            \end{enumerate}
            
            Pour tout $(t,x)\in \R \times \R^d$ considérons le pb de Cauchy suivant:
            
            \begin{equation}
                \begin{cases}
                \frac{dy}{ds}(s) & = b(s,y(s)) \\ 
                y(t) & = x  
                \end{cases}\label{eq:1}
            \end{equation}
            
            
            \begin{proposition}\label{prop:1.2.1}
                Le problème de Cauchy $\eqref{eq:1}$ admet une unique solution $s\mapsto y(s) \in \C^1(\ff{0,T})$
            \end{proposition}
            
            \begin{proof}
                ...
            \end{proof}

            \begin{definition}\label{def:1.2.2}
                L'ensemble $\{ (s,y(s)),s\in \R\} \subset \R \times \R^d$ est appelé \defemph{courbe caractéristique} de l'opérateur " $\partial_t + b \cdot \nabla$ " passant par $x$ à l'instant $t$.
            \end{definition}

            \begin{remark}\label{rem:1.2.3}
                On note dans la suite: $X(s,t,x) = y(s)$ l'unique solution de la proposition \ref{prop:1.2.1} de sorte que 
                
            \begin{equation*}
                \begin{cases}
                    \partial_s X(s,t,x) &= b(s,X(s,t,x)) \\
                    X(t,t,x)& = x
                \end{cases}
            \end{equation*}
                
            \end{remark}

            \begin{definition}\label{def:1.2.4}
                L'application $X:\ff{0,T} \times \ff{0,T} \times \R^d \to \R^d$ est appelée \defemph{flot caractéristique} de l'opérateur $\partial_t + b \cdot \nabla$.
            \end{definition}

            \begin{proposition}\label{prop:1.2.5}
                Le flot caractéristique de l'opérateur $\partial_t + b \cdot \nabla$ vérifie les propriétés suivantes:

                \begin{enumerate}
                    \item $\forall t_1,t_2,t_3 \in \ff{0,T}$, on a $X(t_3,t_2,X(t_2,t_1,x)) = X(t_3,t_1,x)$

                    \item $X\in \C^1(\ff{0,T}^2 \times \R^d; \R^d)$

                    \item $\forall i\in \llbracket 1, d \rrbracket$, les dérivées partielles secondes $\partial_s\partial_{x_i}X(s,t,x)$ et $\partial_{x_i}\partial_sX(s,t,x)$ existent pour tout $(s,t,x) \in \ff{0,T}^2 \times \R^d$. Elles vérifient $\partial_s\partial_{x_i}X(s,t,x) = \partial_{x_i}\partial_sX(s,t,x)$ et sont continues.

                    \item $\forall s,t \in \ff{0,T}$, l'application
                    \begin{equation*}
                        \begin{aligned}
                            X(s,t,\cdot):\R^d&\to\R^d \\
                            x &\mapsto X(s,t,x)
                        \end{aligned}
                    \end{equation*}
                    est un $\C^1$-difféomorphisme de $\R^d$ sur lui-même, et on a $X(s,t,\cdot)^{-1} = X(t,s,\cdot)$
                \end{enumerate}
            \end{proposition}

            \begin{proof}
                ...
            \end{proof}

            \begin{theorem}\label{thm:1.2.6}
                Soit $b:\ffo \times \R^d \to \R^d$ vérifiant les hypothèses du début de section et $u_0\in \C^1(\R^d)$. Notons $X$ le flot caractéristique de $\partial_t + b \cdot \nabla$.
                Alors le problème de Cauchy 

                \begin{equation*}
                    \begin{cases}
                    \partial_t u(t,x) + b(t,x) \cdot \nabla u(t,x)  = 0 & (t,x) \in \ooo \times \R^d \\
                    u(0,x)  = u_0(x) &x \in \R^d
                    \end{cases}\label{eq:2}
                \end{equation*}
                
                admet une unique solution $u\in \C^1(\fo{0,T}\times \R^d)$, donnée par la formule suivante:
                \[
                \forall(t,x) \in \fo{0,T} \times \R^d, u(t,x) = u_0(X(0,t,x))
                \]
            \end{theorem}

            \begin{proof}
                ...
            \end{proof}


    \newpage
    \begin{center}
        \section*{Chapitre 2 --- Compléments sur les distributions}\label{sec:CH2}
    \end{center}
    
    \addcontentsline{toc}{section}{Chapitre 2 --- Compléments sur les distributions}
    \setcounter{section}{2}
    \setcounter{subsection}{0}
        \subsection{Suites de distributions}\label{subsec:2.1}

            Nous allons commencer par quelques rappels du cours d'EDP1.
    
            \begin{definition}[Convergence d'une suite de distributions] \label{2.1.1}~
            
                Soit $(T_n) \in \Dp $. On dit que \defemph{\text{$(T_n)$} est convergente} si et seulement si 
                \[
                    \forall \vphi \in \D, (\langle T_n,\vphi \rangle)_n \text{ est convergente.}
                \]
            \end{definition}
    
            \begin{remark}\label{2.1.2}
                On peut alors affirmer que la limite est une distribution (voir plus loin).
            \end{remark}
    
            \begin{theorem}[Borne uniforme sur $\Dp(\R^d)$]\label{thm:2.1.3}~

                Soit $(T_n) \in \Dp(\R^d)$  une suite convergente. Alors 
                \begin{equation*}
                    \begin{aligned}
                        \forall K \subset \R^d \text{ compact},\ \exists C>0,\ \exists N\in \N \text{ tq } \forall \vphi \in \D, \\ 
                        \text{supp}(\vphi) \subset K \ \Rightarrow \ \sup\limits_{n\in\N}\ \n{\langle T_n,\vphi \rangle} \leq C \times P_{N,K}(\vphi)
                    \end{aligned}
                \end{equation*}
            \end{theorem}
    
            \begin{proof}
                Ce résultat est une variante du théorème de Banach-Steinhaus et ne sera pas démontré ici.
            \end{proof}
    
            \begin{remark}\label{rem:2.1.4}
                $P_{N,K}(\vphi)$ est définie au chapitre 0. \\
                Puisque $(T_n)_n$ est supposée convergente, cela implique que $\forall n, \ \n{\langle T_n,\vphi \rangle}$ est bornée.
            \end{remark}
    
            Voici deux conséquences remarquables du théorème précédent:
    
            \begin{corollaire}\label{corol:2.1.5}
                 Soit $(T_n) \in \Dp(\R^d)$ telle que $\forall \vphi \in \D(\R^d)$, la suite $(\langle T_n,\vphi \rangle)_n$ est convergente. Alors il existe $T \in \Dp(\R^d)$ telle que  $\forall \vphi \in \D(\R^d) $ \ $\underset{n \to +\infty}{\lim}\langle T_n,\vphi \rangle = \langle T,\vphi \rangle$. \\
                 On dit que $T_n \stackrel{\Dp}{\longrightarrow} T$.
            \end{corollaire}
    
            \begin{proof}
                ...
            \end{proof}
    
            \begin{corollaire}[Continuité séquentielle du crochet de dualité]\label{corol:2.1.6}~
            
                Soit $(T_n) \in \Dp$ et $(\vphi_n) \in \D$ telles que $T_n \stackrel{\Dp}{\longrightarrow}T$ et $\vphi_n \stackrel{\D}{\longrightarrow}\vphi$. Alors $\langle T_n,\vphi_n\rangle \underset{n \to +\infty}{\longrightarrow} \langle T,\vphi \rangle$.
            \end{corollaire}
    
            \begin{proof}
                ...
            \end{proof}

        \subsection{Familles de distributions paramétrées par un réel}\label{subsec:2.2}
            Motivations à l'introduction de fonctions du temps à valeur dans un EVN ou équipé d'une famille de semi-normes séparantes pour l'étude de certains problèmes d'évolution. \\[3mm]
            On se donne ici un opérateur différentiel à coefficients constants sur $\R \times \R^d$
            \begin{equation*}
                \begin{aligned}
                    p(\partial_t,\partial_x) = \sum_{k=0}^{m} A_k(\partial_x)\partial_t^k
                \end{aligned}
            \end{equation*}
            
            où $A_k(\partial_x) = \sum_{finie}a_{k,\alpha}\partial_k^{\alpha}$ et $a_{k,\alpha} =$ constante $\in \mathbb{C}$. \\
    
            On se donne des fonctions $f,u_0,...,u_{m-1}$ et on cherche une solution $(t,x) \longmapsto u(t,x)$ du problème suivant:

            \begin{equation}
                \begin{cases}
                    p(\partial_t,\partial_x)u &= f,\\
                    u(0,x) &= u_0(x),\\
                    \vdots & \\
                    \partial_t^{m-1}u(0,x) &= u_{m-1}(x).
                \end{cases}\label{eq:3}
            \end{equation}

            il s'agit du $\underline{\text{Problème de Cauchy}}$ pour l'équation $p(\partial_t,\partial_x)u=f$.\\

            Nous verrons dans la suite une méthode de résolution de ce problème pour des équations "usuelles".
            Pour de telles équations, essentiellement issues de la physique, les deux variables $t$ et $x$ jouent des rôles très différents, $t$ représentant le temps ($t \geq 0)$ et $x$ représentant la position ($x \in \R^d$, $d \in \{ 1,2,3\}$).\\[2mm]

            Il est donc intuitivement légitime de "regarder" une fonction $\R \times \R^d \ni (t,x) \longmapsto u(t,x) \in \R$ comme une fonction $\R \ni t \longmapsto \widetilde{u}(t)$ à valeurs dans un espace vectoriels de fonctions définies sur $\R^d$. (C'est-à-dire $\R^d \ni x \longmapsto \widetilde{u}(t)(x) \in \R$).

            Dans un cadre plus général, cela revient formellement à dire, étant donné trois ensembles $X,Y,Z$ , que nous avons un isomorphisme canonique $\phi$ entre les ensembles $\mathscr{F}(X\times Y;Z)$ (fonctions définies sur $X\times Y$ à valeurs dans $Z$) et $\mathscr{F}(X;\mathscr{F}(Y;Z))$. Cet isomorphisme est donné par les formules qui suivent, réciproques l'une de l'autre ($\phi(u)=\widetilde{u}$):

            \begin{enumerate}
                \item Pour $u \in \mathscr{F}(X\times Y;Z)$, on définit $\widetilde{u} \in \mathscr{F}(X;\mathscr{F}(Y;Z))$ par $\widetilde{u}(x) = u(x,\cdot)$.

                \item Pour $\widetilde{u} \in \mathscr{F}(X;\mathscr{F}(Y;Z))$, on définit $u \in \mathscr{F}(X\times Y;Z)$ par $u(x,y)=\widetilde{u}(x)(y)$.
            \end{enumerate}

            Pour donner un sens aux données initiales pour le problème de Cauchy \eqref{eq:3}, il peut être intéressant de chercher les solutions dans des espaces où les fonctions sont particulièrement régulières en temps, afin que $u(0,x),..,\partial_t^{m-1}u(0,x)$ soient définies.

            On utilisera de manière indifférenciée les deux notations $u(t,\cdot)$ et $u(t)(\cdot)$\\
            \begin{definition}\label{2.2.1}~
            
                On considère une application $\vphi:\R \longrightarrow \D(\R^d)$. 
                
                On dit que \defemph{\text{$\vphi \in \C^0(\R;\D(\R^d))$}} si et seulement si, 
                \begin{equation*}
                    \begin{aligned}
                        \forall s \in \mathbb{R}, \  \forall (s_n)_n \subset \mathbb{R}, \ s_n \to s \Rightarrow \ \varphi(s_n) \to \varphi(s) \text{ dans } \mathcal{D}(\mathbb{R}^d).
                    \end{aligned}
                \end{equation*}
            \end{definition}

            \begin{rappel}
                Soit $(u_n)_n \subset \D(\R^d)$,
                \begin{equation*}
                    \begin{aligned}
                        u_n \stackrel{\D}{\longrightarrow}0 \Leftrightarrow \ \exists K \subset \R^d \text{ compact,} \ \forall n \in \N, \ \text{supp}(u_n) \subset K \text{et } \forall j \in \N, \ P_{j,K}(u_n) \longrightarrow 0.
                    \end{aligned}
                \end{equation*}
            \end{rappel}

            \begin{proposition}\label{prop:2.2.2}
                Soit $s \mapsto \vphi(s) \in \C^0(\R;\D(\R^d))$.\\
                Alors $\forall S \subset \R \text{ compact, } \exists K \subset \R^d$ compact tels que
                \begin{equation*}
                    \begin{aligned}
                        \forall s \in S, \ \text{supp}(\vphi(s)) \subset K \text{ et } \forall j \in N, \text{ l'application } S \ni s \mapsto P_{j,K}(\vphi(s)) \text{ est continue.}
                    \end{aligned}
                \end{equation*}
            \end{proposition}

            \begin{proof}
                ...
            \end{proof}

            \begin{definition}\label{2.2.3}
                On considère une application $\R \ni s \mapsto T(s)\in \Dp(\R^d)$.
                
                \begin{enumerate}
                    \item On dit que \defemph{$s \mapsto T(s) \in \C^0(\R;\Dp(\R^d))$} si et seulement si,
                    \begin{equation*}
                        \begin{aligned}
                            \forall \vphi \in \D(\R^d), \text{ l'application } \R \ni s\ \longmapsto \ \langle T(s),\vphi \rangle_{\Dp,\D} \text{ est continue.}
                        \end{aligned}
                    \end{equation*}

                    \item On dit que \defemph{\text{$s \mapsto T(s) \in \C^k(\R;\Dp(\R^d))$}} si et seulement si,
                    \begin{equation*}
                        \begin{aligned}
                            \forall \vphi \in \D(\R^d), \text{ l'application } \R \ni s \longmapsto {\langle T(s),\vphi \rangle}_{\Dp,\D} \text{ est de classe } \C^k(\R).
                        \end{aligned}
                    \end{equation*}
                \end{enumerate}
            \end{definition}

            Il existe une variante du théorème \ref{thm:2.1.3} adaptée aux familles de distributions paramétrées:
            
            \begin{theorem}[Principe de la borne uniforme]\label{thm:2.2.4}~
            
                Soit $S \subset \R$ compact et $s \mapsto T(s) \in \C^0(\R;\Dp(\R^d))$. Alors 
                \begin{equation*}
                    \begin{aligned}
                        \forall K \subset \R^d \text{ compact, } \exists &C>0, \ \exists N \in N, \ \forall s \in S,\ \forall \vphi \in \D(\R^d) \text{ tel que}\\
                        \text{supp}(\vphi) \subset K &\Rightarrow \ \n{\langle T(s),\vphi \rangle} \leq C*P_{N,K}(\vphi)
                    \end{aligned}
                \end{equation*}
            \end{theorem}

            \begin{proof}
                Tout comme le théorème \ref{thm:2.1.3}, ce résultat ne sera pas démontré.
            \end{proof}
            
            \begin{proposition}\label{prop:2.2.5}
                Soit $T \in \C^k(\R,\Dp)$. \\
                Alors $\forall l \in \{0,..,k \}, \ \forall s \in \R,$ il existe une distribution, notée $T^{(l)}(s)$, telle que l'application $s \mapsto T^{(l)}(s) \in \C^{k-l}(\R,\Dp(\R^d))$ et telle que $\forall s_0 \in \R, \forall \vphi \in \D,$
                \begin{equation*}
                    \begin{aligned}
                        {\left( \frac{d^{(l)}}{ds} \langle T(s), \varphi \rangle \right)}_{|s = s_0} = \left\langle T^{(l)}(s_0), \varphi \right\rangle.
                    \end{aligned}
                \end{equation*}
            \end{proposition}

            \begin{proof}
                \dots
            \end{proof}

            \begin{proposition}\label{prop:2.2.6}
                Soit $T \in \C^0(\R,\Dp)$ et $\psi \in \D(\R \times \R^d)$. Alors l'application
                \begin{equation*}
                    \begin{aligned}
                        s \longmapsto {\langle T(s),\ \psi(s,\cdot) \rangle}
                    \end{aligned}
                \end{equation*}
                est continue.
            \end{proposition}

            \begin{proof}
                ...
            \end{proof}

            \begin{proposition}\label{prop:2.2.7}
                Soit $T \in \C^1(\R,\Dp)$ et $\psi \in \D(\R \times \R^d)$. Alors l'application
                \begin{equation*}
                    \begin{aligned}
                        s \longmapsto \ {\langle T(s),\ \psi(s,\cdot) \rangle}
                    \end{aligned}
                \end{equation*}
                est $\C^1$ et on a 
                \begin{equation*}
                    \begin{aligned}
                        \frac{d}{ds}{\langle T(s),\psi(s,\cdot) \rangle} \ = \ {\langle T^{(1)}(s),\psi(s,\cdot)\rangle} + {\langle T(s),\p_1 \psi(s,\cdot)\rangle}
                    \end{aligned}
                \end{equation*}
            \end{proposition}

            \begin{proof}
                ...
            \end{proof}

            \begin{proposition}\label{prop:2.2.8}
                Soit $T \in \C^0(\R,\Dp(\R^d))$. On pose, pour $\psi \in \D(\R \times \R^d)$,
                \begin{equation*}
                    \begin{aligned}
                        {\langle \widetilde{T},\psi\rangle} = \int_{\R^d} {\langle T(s),\psi(s,\cdot) \rangle}_{\Dp} ds
                    \end{aligned}
                \end{equation*}
                Alors $\widetilde{T} \in \Dp(\R \times \R^d)$.
            \end{proposition}

            \begin{proof}
                ...
            \end{proof}

            \begin{proposition}\label{prop:2.2.9}
                Soit $T \in \C^0(\R,\Dp)$. Il existe une distribution, notée $\bigintss_0^t T(s)ds$, définie $\forall t \in \R$ par:
                \begin{equation*}
                    \begin{aligned}
                        \forall \vphi \in \D, \ {\langle \bigintssss_0^t T(s)ds,\vphi\rangle} \ = \bigintssss_0^t {\langle T(s),\vphi\rangle}\ ds
                    \end{aligned}
                \end{equation*}
            \end{proposition}

            \begin{proof}
                ...
            \end{proof}

            \begin{theorem}[Théorème Fondamental de l'analyse]\  \label{thm:2.2.10}
                \begin{enumerate}
                    \item Soit $f \in \C^0(\R,\Dp)$. La fonction $F$ définie par
                        \begin{equation*}
                            \begin{aligned}
                                F(t) = \int_0^t f(s)ds
                            \end{aligned}
                        \end{equation*}
                        est $\C^1(\R,\Dp)$, et $F'(t)=f(t)$.

                    \item Soit $F\in\C^1(\R,\Dp)$. On a: $\forall a \in \R$,
                        \begin{equation*}
                            \begin{aligned}
                                F(t) = F(a) + \int_a^t F'(s)ds
                            \end{aligned}
                        \end{equation*}
                \end{enumerate}
            \end{theorem}

            \begin{proof}
                ...
            \end{proof}

            \begin{noremark}
                Le problème: Trouver $F \in \C^1(\R,\Dp)$ vérifiant 
                \begin{equation}
                    \begin{cases}\label{eq:4}
                        F'(t)&=\alpha(t),\\
                        F(a) &= \beta
                    \end{cases}
                \end{equation}
                avec $\alpha \in \C^0(\R,\Dp)$, $\beta \in \Dp$ possède une unique solution.

                En effet, cette solution est donnée par le 2) du théorème \ref{thm:2.2.10}. 

                Soit $F\in \C^1(\R,\Dp)$ solution de \eqref{eq:4}, alors nécessairement, $\forall t \in \R,$
                \begin{equation*}
                    F(t)=F(a)+\int_a^tF'(s)ds
                \end{equation*}
                d'où
                \begin{equation*}
                    F(t) = \beta + \int_a^t \alpha (s)ds
                \end{equation*}

                Réciproquement, pour $F(t) = \beta + \int_a^t \alpha (s)ds$, on a 
                \begin{equation*}
                    F(a)=\beta+\underbrace{\int_a^a \alpha(s)ds}_{\text{\shortstack{= 0 par \\ 
                    théorème \ref{thm:2.2.10}}}}
                \end{equation*}
                et $h>0, \ F'(t) = \lim\limits_{h \to 0} \frac{F(t+h)-F(t)}{h}=\left( \bigintsss_0^t  \alpha(s)ds \right)'=\alpha(t).$ 
            \end{noremark}

            \begin{corollaire}\label{corol:2.2.11}
                Soit $a \in \C^{\infty}(\R)$. Le problème:
                \begin{equation*}
                    \left| 
                    \begin{array}{l}
                        \text{Trouver } T \in \mathscr{C}^1(\mathbb{R}; \mathscr{D}') \text{ solution de } \\
                        \begin{cases}
                            T'(s) + a(x)T(s) = \alpha(s), \\
                            T(0) = \beta,
                        \end{cases} \\[10pt]
                        \text{avec } \alpha \in \mathscr{C}^0(\mathbb{R}; \mathscr{D}') \text{ et } \beta \in \mathscr{D}',
                    \end{array}
                    \right.
                \end{equation*}
                admet une unique solution donnée par: 
                \begin{equation*}
                    T(t) = e^{-ta(x)}\beta + \int_0^t e^{-(t-s)a(x)} \alpha(s)ds
                \end{equation*}
            \end{corollaire}

            \begin{proof}
                ...
            \end{proof}

        \newpage
        \subsection{Distributions tempérées, Fourier, Sobolev}\label{subsec:3}
            On notera dans la suite $\cxi = \eqxi^{1/2}$

            On a donc $\cxi^s = \eqxi^{s/2}$ et on note dans la suite $L_s^2(\R^d) := L^2(\R^d;\cxi^{2s}d\xi)$

            \begin{rappel}
            \begin{equation*}
                \begin{aligned}
                    H^k(\R^d) &= \{ u \in L^2(\R^d), \p^{\alpha}u \in L^2(\R^d), \forall \n{\alpha} \leq k \} \text{ pour } k\in\N,\\[2mm]
                    H^s(\R^d) &= \{ u \in \Sp(\R^d), \cxi^s \, \widehat{u} \in L^2(\R^d) \} \text{ pour } s \in \R,\\[2mm]
                    &\begin{rcases}
                        \begin{aligned}
                            \nn{u}_{H^k} &= \left(\sum\limits_{\n{\alpha}\leq k} \nn{\p^{\alpha}u}_{L^2}^2\right)^{1/2}\\[2mm]
                            \nn{u}_{H^s} &= \left(\int_{\R^d}\cxi^{2s} \ \n{\widehat{u}(\xi)}^2 \ d\xi\right)^{1/2} \ 
                        \end{aligned}
                    \end{rcases}
                    \text{ équivalence}
                \end{aligned}
            \end{equation*}

            Lorsque $s=k \in \N, \nn{u}_{H^k}$ est équivalente à $\sum\limits_{\n{\alpha}\leq k} \nn{\p^{\alpha}u}_{L^2}$.

            Soient $s,t \in \R$ et $\alpha \in \N^d$. On a les propriétés suivantes:
            \begin{enumerate}
                \item L'injection $\Sp(\R^d) \hookrightarrow H^s(\R^d)$ est dense.
                \item $s\leq t \implies H^t \subset H^s$.
                \item L'application $\p^{\alpha}\colon H^s \rightarrow H^{s-\n{\alpha}}$ est continue.
            \end{enumerate}
            \end{rappel}
    
            \begin{proposition}\label{prop:2.3.1}
                $F$ réalise une isométrie entre $H^s(\R^d)$ et $ L^2_s(\R^d)$.
            \end{proposition}
    
            \begin{proof}
                ...
            \end{proof}
    
            \begin{definition}[Transformée de Fourier partielle]\label{def:2.3.2}~
            
                Soit $u \in \scrS(\R \times \R^d)$, avec la notation de variables $(t,x) \in \R \times \R^d$. La \defemph{transformée de Fourier partielle par rapport à \text{$x$}} de $u$ est la fonction
                \begin{equation*}
                    \widetilde{u}(t,\xi) = \scrF_{x \to \xi}(u(t,x))
                \end{equation*}
                par 
                \begin{equation*}
                    \widetilde{u}(t,\xi) = \frac{1}{(2\pi)^\frac{d}{2}} \bigintsss_{\R^d}u(t,x) e^{-ix \cdot \  \xi}dx
                \end{equation*}
            \end{definition}
    
            \begin{noremark}
                On note cette transformée partielle à l'aide d'un "$\underset{\cdot}{\widetilde{~~~}}$" pour la différencier de la transformée "usuelle" notée avec le " $\underset{\cdot}{\widehat{~~~}}$ "  .
            \end{noremark}~

            En raisonnant à $t \in \R$ fixé, on a :

            \begin{proposition}\label{prop:2.3.3}
                La transformée de Fourier partielle $\scrF_{x\to \xi}$ est un isomorphisme de $\scrS(\R 
                \times \R^d)$, d'inverse $\scrF^{-1}_{x \to \xi}$:
                \begin{equation*}
                    \scrF^{-1}_{x \to \xi}(u(t,\xi)) = (2\pi)^{-\frac{d}{2}} \int_{\R^d} u(t,\xi) e^{ix\cdot \xi} d\xi
                \end{equation*}
                De plus, la transformée de Fourier partielle est continue:
                \begin{equation*}
                    \begin{aligned}
                        \forall k \in \N^*, \ \exists C_k>0, \ \forall u \in \scrS(\R \times \R^d), \\
                        N_k(\widetilde{u}) \leq C_k *N_{k+d+1}(u)
                    \end{aligned}
                \end{equation*}
            \end{proposition}

            \begin{proof}
                Similaire à celle du chapitre 0.
            \end{proof}

            \newpage
            
            La transformée de Fourier partielle s'étend à $\Sp\RRd$ par dualité: pour $T\in \Sp\RRd$,
            \begin{equation*}
                    \widetilde{T} = \scrF_{x \to \xi}(T)
            \end{equation*}
            est définie par
            \begin{equation*}
                \forall u \in \scrS\RRd, \ \langle \scrF_{x \to \xi}T,u \rangle = \langle T, \scrF_{x \to \xi}\ u \rangle
            \end{equation*}
            On a alors le résultat suivant:
        
            \begin{proposition}\label{prop:2.3.4}
                La transformée de Fourier partielle est un isomorpshime de $\Sp\RRd$, d'inverse 
                \begin{equation*}
                    \scrF_{x \to \xi}^{-1} = (2\pi)^{-\frac{d}{2}} \ \widecheck{\scrF}_{x \to \xi}
                \end{equation*}
                De plus, si $T_n \to T$ dans $\Sp\RRd$, alors $\widetilde{T_n} \to \widetilde{T}$ dans $\Sp\RRd$.
            \end{proposition}

            \begin{proof}
                On procède par transposition, comme d'habitude pour prolonger de $\scrS$ vers $\Sp$.
            \end{proof}

            \begin{remark}\label{rem:2.3.5}
                Les règles de calcul "usuelles" s'adaptent aisement:
                \begin{equation*}
                    \widetilde{\p_x^{\alpha}T} = \xi^{\alpha}\widetilde{T}\ \ \text{ et     } \ \ \widetilde{x^{\alpha} T} = (-\p_{\xi})^{\alpha}\widetilde{T}
                \end{equation*}
                \underline{La formule de permutation $\widetilde{\p_t T} = \p_t \widetilde{T}$ devra être justifiée au cas par cas}.
            \end{remark}

            \begin{remark}\label{rem:2.3.6}
                Lorsque $u \in \scrS\RRd$ (ou $\Sp\RRd$) est vue comme une fonction dépendant de $t\in \R$ à valeurs dans $\scrS(\R^d)$, on pourra noter de manière équivalente:
                \begin{equation*}
                    \widehat{u(t)}(\xi) = \widetilde{u}(t,\xi)
                \end{equation*}
                (puisque $u(t) \in \scrS(\R^d), \ \widehat{u(t)} \in \scrS(\R^d)$)
            \end{remark}

        \subsection{Fonctions régulières à valeurs dans un Sobolev}\label{subsec:4}
            On considère une fonction $\R \in t \mapsto u(t) \in H^s(\R^d)$.

            \begin{definition}\label{def:2.4.1}
                Soit $\R \in t \mapsto u(t) \in H^s(\R^d)$. On dit que \defemph{\text{$u \in \C^0(\R,H^s(\R^d))$}} si et seulement si 
                \begin{equation*}
                    \forall t \in \R, \ \forall (t_n) \text{ telle que } t_n \to t, \text{ on a } u(t_n) \stackrel{H^s}{\longrightarrow} u(t)
                \end{equation*}
            \end{definition}

            \begin{definition}\label{def:2.4.2}
                On dit que l'application $\rho \mapsto u(\rho)$ est \defemph{dérivable en $t$} à valeurs dans $H^s(\R^d)$ et \defemph{de dérivée} $u'(t)$ si et seulement si
                \begin{equation*}
                    \forall h>0, u(t+h) = u(t)+hu'(t)+h\phi(h)
                \end{equation*}
                avec $\phi \in \C^0(\R,H^s(\R^d))$ telle que $\phi(0)=0$.
            \end{definition}

            \begin{definition}\label{def:2.4.3}
                On dit que \defemph{$u \in \C^1(\R,H^s(\R^d))$} si et seulement si
                \begin{equation*}
                    t \longmapsto u(t) \text{ est dérivable en tout point }t
                \end{equation*}
                avec $\forall t\in \R, u'(t) \in H^s(\R^d)$ et $t\longmapsto u'(t) \in \C^0(\R,H^s(\R^d))$.
            \end{definition}

            Pour $u \in \C^k(\R,H^s(\R^d))$, on définit $T_u$ par: $\forall \vphi \in \D\RRd$,
            \begin{equation*}
                \langle T_u,\vphi \rangle = \int_{\R^d} \langle u(t),\vphi(t,\cdot) \ \rangle dt
            \end{equation*}

            \begin{proposition}\label{prop:2.4.4}
                Pour tout $u \in \C^k(\R,H^s(\R^d))$, on a $T_u \in \Dp\RRd$. De plus l'application linéaire $u \longmapsto T_u$ est injective.
            \end{proposition}

            \begin{proof}
                ...
            \end{proof}

            \begin{proposition}\label{prop:2.4.5}
                $\scrF_{x \to \xi}$ est une isométrie de $\C^0(\R,H^s(\R^d))$ sur $\C^0(\R,L^2_s(\R^d))$
            \end{proposition}

            \begin{proof}
                C'est immédiat à partir de la proposition \ref{prop:2.3.1} et la définition de $\scrF_{x \to \xi}$.
            \end{proof}

            \begin{proposition}\label{2.4.6}
                Soit $u\colon \R \times \R^d \longrightarrow \mathbb{C}$ mesurable, et $a,b \in \R$, tels que $a<b$. On suppose que :
                \begin{enumerate}
                    \item Il existe un ensemble négligeable $N$ tel que pour tout $x \in \R^d\setminus N$, $t \longmapsto u(t,x) $ est continue.

                    \item Il existe $g \in L^2_s(\R^d)$ positive telle que: 
                    \begin{equation*}
                        \sup\limits_{t\in \ff{a,b}}\n{u(t,\cdot)} \leq g
                    \end{equation*}
                \end{enumerate}
                Alors on a $u\in \C^0(\ff{a,b}; \ L^2_s(\R^d))$
            \end{proposition}

            \begin{proof}
                ...
            \end{proof}

            \begin{proposition}\label{prop:2.4.7}
                Soit $u\colon \R \times \R^d \longrightarrow \mathbb{C}$ mesurable, et $a,b \in \R$, tels que $a<b$. On suppose que :
                \begin{enumerate}
                    \item Il existe un ensemble négligeable $N$ tel que pour tout $x \in \R^d\setminus N$, $t \longmapsto u(t,x) $ est de régularité $\C^1$.

                    \item Il existe $g \in L^2_s(\R^d)$ positive telle que: 
                    \begin{equation*}
                        \sup\limits_{t\in \ff{a,b}}\n{u(t,\cdot)} \leq g
                    \end{equation*}

                    \item Il existe $h \in L^2_s(\R^d)$ positive telle que: 
                    \begin{equation*}
                        \sup\limits_{t\in \ff{a,b}}\n{\p_tu(t,\cdot)} \leq h
                    \end{equation*}
                \end{enumerate}
                Alors on a $u\in \C^1(\ff{a,b}; \ L^2_s(\R^d))$
                
            \end{proposition}

            \begin{proof}
                ...
            \end{proof}


    \newpage
    \begin{center}
        \section*{Chapitre 3 --- Equation de la chaleur}\label{sec:CH3}
    \end{center}
    
    \addcontentsline{toc}{section}{Chapitre 3 --- Equation de la chaleur}
    \setcounter{section}{3}
    \setcounter{subsection}{0}

        On étudie le problème de Cauchy suivant: 
        \begin{equation}\label{eq:5}
            \left| 
            \begin{array}{l}
                \text{Trouver } u:\R \times \R^d \ni (t,x) \longmapsto u(t,x) \in \R \text{ telle que } \\
                \begin{cases}
                    \p_tu- \Delta u = 0, \text{ sur } \R \times \R^d\\
                    u(0,\cdot) = u_0,
                \end{cases} \\[10pt]
            \end{array}
            \right.
        \end{equation}
        C'est l'équation de la chaleur qui modélise des phénomènes d'évolution: diffusion de la chaleur, répartition de substances chimiques \dots

        Nous allons préciser dans la suite la régularité de la donnée initiale $u_0$ et ses conséquences sur la régularité de la solution.\\[2mm]

        \subsection{Donnée initiale dans $\Sp(\R^d)$}\label{subsec:3.1} 
            \begin{theorem}[Solution dans $\C^1(\R,\Sp(\R^d))$]\label{thm:3.1.1}~
            
                On considère le problème de Cauchy \eqref{eq:5} avec $u_0 \in \Sp(\R^d)$.
                Alors \eqref{eq:5} admet une unique solution $u \in \C^1(\R_+;\Sp(\R^d))$ donnée par:
                \begin{equation*}
                    \forall t \in \R_+, u(t) = \scrF^{-1}(\widehat{u}(t))
                \end{equation*}
                avec
                \begin{equation*}
                    \widehat{u}(t) = e^{-t\n{\xi}^2}\widehat{u_0}
                \end{equation*}
            \end{theorem}

            \begin{proof}
                ...
            \end{proof}

        \subsection{Donnée initiale dans $\Hs$.}\label{subsec:3.2}
            Nous imposons dans cette section une régularité beaucoup plus forte sur $u_0$.

            \begin{theorem}[Solution dans $\C^1(\R_+;H^{s-2}(\R^d))$] \label{thm:3.2.1}~
            
                On considère me pb de Cauchy \eqref{eq:5} avec $u_0 \in \Hs, \ s \in \R$. Alors \eqref{eq:5} admet une unique solution 
                \begin{equation*}
                    u \in \C^0(\R_+;H^s) \cap \C^1(\R_+;H^{s-2})
                \end{equation*}
            \end{theorem}

            \begin{remark}\label{rem:3.2.2}
                Nous allons bien entendu utiliser le fait que $\Hs \hookrightarrow \Sp(\R^d)$ afin d'éviter de tout redémontrer.
            \end{remark}

            \begin{proof}
                ...
            \end{proof}

        \subsection{Quelques propriétés qualitatives}\label{subsec:3.3}
            \begin{theorem}[Noyau de la chaleur] \label{thm:3.3.1}~
            
                Soit $t>0, \ u_0\in \Sp(\R^d)$. Alors
                \begin{equation*}
                    u(t) = K_t \ast u_0
                \end{equation*}
                est solution de \eqref{eq:5} avec $K_t$ défini ainsi: $\forall x \in \R^d$,
                \begin{equation*}
                    K_t(x)= (4t\pi)^{-\frac{d}{2}}e^{-\frac{\n{x}^2}{4t}} \in \scrS (\R^d)
                \end{equation*}
            \end{theorem}

            \begin{proof}
                ...
            \end{proof}

            \begin{remark}\label{rem:3.3.2}
                $K_t$ est parfois aussi appelé "Noyau de la chaleur".
            \end{remark}

            \begin{remark}\label{rem:3.3.3}
                On observe de manière évidente que $\forall t>0, \ K_t \in \scrS(\R^d)$.

                Ainsi, si $u_0 \in \scrS(\R^d), \ u(t) \in \scrS(\R^d)$.
            \end{remark}

            \begin{proposition}[Régularité parabolique]\ \label{prop:3.3.4}
            
                On suppose $u_0 \in \C^0_b(\R^d)$. Alors la fonction $(t,x) \mapsto  u(t,x)$ définie par 
                \begin{equation*}
                    u(t,x) = \frac{1}{(4t\pi)^\frac{d}{2}} \bigintsss_{\R^d} e^{-\frac{\n{x-y}^2}{4t}}u_0(y)\ dy
                \end{equation*}
                est $\C^{\infty}(\R^*_+ \times \R^d)$ et vérifie $\p_tu-\Delta u=0$ sur $\R^*_+ \times \R^d$.

                De plus, on peut la prolonger par continuité à $\R_+ \times \R^d$ en posant 
                \begin{equation*}
                    u(0,x) = u_0(x) \ \forall x \in \R^d
                \end{equation*}
                de sorte que $u \in \C^0_b(\R_+ \times \R^d)$ et vérifie \eqref{eq:5}.
            \end{proposition}

            \begin{proof}
                ...
            \end{proof}

            \begin{proposition}[Principe du maximum]\label{3.3.5}~ 
            
                Soit $u \in \C^{\infty}(\R^*_+ \times \R^d)\cap \C_b(\R_+\times \R^d)$ solution de \eqref{eq:5}. Alors 
                \begin{equation*}
                    \sup\limits_{(t,x)\in \R_+\times \R^d} u(t,x) = \sup\limits_{x\in \R^d}u_0(x)
                \end{equation*}
            \end{proposition}

            \begin{proof}
                ...
            \end{proof}

            \begin{remark}\label{rem:3.3.6}                
                Il résulte également du principe du maximum si $u_0$ est non nulle, alors \\ 
                $u(t,x)>0$ sur $\R^*_+ \times \R^d$. Cette propriété est particulièrement frappante lorsque $u_0 \in \D$ par exemple: une partie de la quantité étudiée s'est immédiatement propagée à l'infini. On parle parfois de "vitesse de propagation infinie".
            \end{remark}

            \begin{definition}[Semi-groupe de la chaleur]\label{def:3.3.7}~

                Pour tout $t\geq 0$, on définit une application linéaire 
                \begin{equation*}
                    e^{t\triangle}\colon L^2 \ni u_0 \longmapsto e^{t\triangle}u_0 \in L^2
                \end{equation*}
                telle que $\forall x \in \R^d \ e^{t\triangle}u_0 = u(t,\cdot)$ la solution du problème de Cauchy homogène de donnée initiale $u_0$. 

                La famille $(e^{t\triangle})_{t \geq0}$ est appelée "semi-groupe de la chaleur".
            \end{definition}

            \begin{remark}\label{3.3.8}
                $e^{t\triangle}$ est une notation et ne correspond pas à une somme de série...
            \end{remark}

            Cette notation permet de souligner l'analogie entre le semi-groupe de la chaleur et la formule $u=e^{-tA}u_0$ donnant la solution du système différentiel linéaire 
            \begin{equation*}
                \begin{cases}
                    \dot{u} + Au &=0,\\
                    u(0) &= u_0
                \end{cases}
            \end{equation*}
            d'inconnue $t\mapsto u(t) \in \R^n$ et $A$ matrice carrée de taille $n\times n$.

            \begin{proposition}[Propriétés de $e^{t\triangle}$]\label{prop:3.3.9}~

                Le semi-groupe de la chaleur $\et$ vérifie les propriétés suivantes:
                \begin{enumerate}
                    \item $\forall t \geq0,$ on a ${\nn{\et u_0}}_{L^2}\leq {\nn{u_0}}_{L^2}$

                    \item $\forall t,s\geq 0$, on a $e^{(t+s)\triangle} = \et e^{s\triangle}$

                    \item Pour tout $u_0 \in L^2$, la fonction  $\R_+ \ni t \mapsto \et u_0 \in L^2$ est continue.
                \end{enumerate}
            \end{proposition}

            \begin{proof}
                ...
            \end{proof}

            \begin{proposition}[Comportement en temps long]\label{3.3.10}~

                Soit $u_0 \in \Hs$, on a
                \begin{equation*}
                    {\nn{K_t \ast u_0}}_{H^s} \underset{t \to +\infty}{\longrightarrow} 0
                \end{equation*}
                Si de plus on a $u_0 \in L^1(\R^d)$, alors $\forall x \in \R^d$, on a
                \begin{equation*}
                    \n{K_t \ast u_0(x) - (4t\pi)^{-\frac{d}{2}} \bigintsss_{\R^d} u_0(y)\ dy} = o(t^{-\frac{d}{2}})
                \end{equation*}
                lorsque $t \to +\infty$.
            \end{proposition}

            \begin{proof}
                ...
            \end{proof}

        \subsection{Donnée initiale dans $\Sp$: Cas non-homogène}\label{subsec:3.4}

             On considère le problème de Cauchy suivant (cas non-homogène)
                \begin{equation}\label{eq:6}
                    \begin{cases}
                        \p_tu- \Delta u = f, \text{ sur } \R \times \R^d\\
                        u(0,\cdot) = u_0,
                    \end{cases}
                \end{equation}

            \begin{theorem}[Solution dans $\C^1(\R;\Sp(\R^d))$]\label{thm:3.4.1}~

                Soit $f \in \C^0(\R_+;\Sp(\R^d))$ et $\ u_0 \in \Sp(\R^d)$.
                
                Alors \eqref{eq:6} admet une unique solution $u \in \C^1(\R_+;\Sp(\R^d))$ donnée par:
                \begin{equation*}
                     \forall t \in \R_+, \ u(t)=\scrF^{-1}(\widehat{u}(t))
                \end{equation*}
                et
                \begin{equation*}
                    \widehat{u}(t) = e^{-t\n{\xi}^2}\widehat{u_0} + \bigintsss_0^t e^{-(t-s)\n{\xi}^2}\widehat{f}(s) \ ds
                \end{equation*}
            \end{theorem}

            \begin{proof}
                ...
            \end{proof}

        \subsection{Donnée initiale dans $H^s$: Cas non-homogène}\label{subsec:3.5}

            \begin{theorem}\label{3.5.1}~
            
                Soit $s \in \R, \ u_0 \in \Hs$ et $f\in \C^0(\R_+;\Hs)$. Alors la fonction
                \begin{equation*}
                    t \mapsto u(t) = \et u_0 + \bigintsss_0^t e^{(t-s)\triangle}f(s,\cdot) \ ds
                \end{equation*}
                est $\C^0(\R_+;\Hs)\cap \C^1(\R_+;H^{s-2}(\R^d))$. De plus, $u$ est l'unique solution de \eqref{eq:6} dans cet espace.
            \end{theorem}

            \begin{proof}
                ...
            \end{proof}
    ~\\ ~\\
    La partie sur l'électromagnétisme ne sera pas explicitée dans ce papier.
\end{document}
